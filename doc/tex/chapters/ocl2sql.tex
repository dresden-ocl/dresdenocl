\chapter{SQL Code Generation}
\label{chapter:ocl2sql}

\begin{flushright}
\textit{Chapter written by Bj�rn Freitag}
\end{flushright}

This chapter describes how the SQL Code Generator \keyword{OCL2SQL} provided 
with Dresden OCL can be used. A general introduction into Dresden OCL is
provided in Chapter~\ref{chapter:introduction}.

The SQL Code Generator is able to generate code of the following SQL
dialects (target languages):
\begin{itemize}
   \item Standard SQL 2008
  \item PostgreSQL 9
  \item Oracle SQL 11g
  \item MySQL 5.1 and 5.5
\end{itemize}


\section{Code Generator Preparation}

In the following, the \keyword{University Example} is used
(cf. Figure~\ref{pic:example:university01}) to explain SQL code generation with
Dresden OCL. To import the University Example into the Eclipse workspace we have
to use the wizard \emph{File -> New -> Other\ldots -> Dresden OCL Examples ->
University Example (UML)}. Afterwards, the project
\model{tudresden.ocl20.pivot.\linebreak[0]examp\-les.\linebreak[0]university}
should be available within the workspace.


The project 
provides a model file which contains the university class diagram (the model file is located at 
\model{model/university.uml}) and the constraint file (located at
\model{con\-straints/\linebreak[0]uni\-ver\-si\-ty\linebreak[0].ocl}) we want to generate code for. One
invariant is shown in Listing~\ref{lst:codegen:universityInvariant}. It is contained in the
constraint file we want to generate code for. The invariant declares, that the
\model{grade} of any \model{Person}'s superviser must be greater thant its own
\model{grade}.


\begin{figure}[!t]
	\centering
	\includegraphics[width=0.85\linewidth]{figures/examples/university01}
	\caption{the UML diagram of the university example.}
	\label{pic:example:university01}
\end{figure}

\begin{figure}[!htbp]
  \lstset{
    language=OCL
  }
  \begin{lstlisting}[caption={a simple invariant.}, captionpos=b, label=lst:codegen:universityInvariant]
context Person
inv tudOclInv1: self.supervisor.grade.value > self.grade.value
  \end{lstlisting}
\end{figure}

\section{Code Generation}
To prepare the code generation we have to import the model 
\model{model/university.uml} into the \eclipse{Model Browser}. We use the model
import wizard of Dresden OCL to import the model. This procedure is explained in
Chapter~\ref{chapter:introduction}. Afterwards, we have to open the constraint
file \model{con\-straints/\linebreak[0]university.ocl}. The imported model
should look within the \eclipse{Model Browser} as illustrated in
Figure~\ref{pic:codegen:modelBrowserSQL}. Now we can start the code generation.
By selecting the item \eclipse{Generate SQL Code} of the menu \eclipse{Dresden
OCL} the SQL code generation can be started.

\begin{figure}[!b]
	\centering
	\includegraphics[width=0.45\linewidth]{figures/codegen/modelBrowserSQL}
	\caption{The Model Browser showing the university model and its constraints.}
	\label{pic:codegen:modelBrowserSQL}
\end{figure}


\subsection{Selecting a Model}
At first, a wizard is opened and we have to select a model for code generation
(cf. Fig.~\ref{pic:codegen:codegen01SQL}). We select the \model{university.uml}
model and click the \eclipse{Next} button.

\begin{figure}[!p]
	\centering
	\includegraphics[width=1.0\linewidth]{figures/codegen/codegen01SQL}
	\caption{The first step: Selecting a model for code generation.}
	\label{pic:codegen:codegen01SQL}

	\vspace{2.0em}
	\centering
	\includegraphics[width=1.0\linewidth]{figures/codegen/codegen02SQL}
	\caption{The second step: Selecting constraints for code generation.}
	\label{pic:codegen:codegen02SQL}
\end{figure}


\subsection{Selecting Constraints}
As a second step we have to select the constraints we want to
generate code for. We only select the constraints \model{inv: oclInv1} and
\model{inv: oclInv2}. After the selection of the constraints we click on the
\eclipse{Next} button (cf. Fig.~\ref{pic:codegen:codegen02SQL}).

\begin{figure}[!p]
	\centering
	\includegraphics[width=1.0\linewidth]{figures/codegen/codegen03SQL}
	\caption{The third Step: Selecting a target directory for the generated code.}
	\label{pic:codegen:codegen03SQL}

	\vspace{2.0em}
	
	\centering
	\includegraphics[width=1.0\linewidth]{figures/codegen/codegen04SQL}
	\caption{The fourth Step: General Settings for the code generation.}
	\label{pic:codegen:codegen04SQL}
\end{figure}


\subsection{Selecting a target directory}
Now, we have to choose the directory where the generated code will be stored.
We select the folder \model{sql} in this project (which is 
\model{tudresden.\linebreak[0]ocl20.pivot.examples.university/\linebreak[0]sql})
(cf. Fig.~\ref{pic:codegen:codegen03SQL}). Then, we click the \eclipse{Next}
button.

 
\subsection{General settings}

On the following page of the wizard we can specify general settings for the code
generation (cf. Fig.~\ref{pic:codegen:codegen04SQL}). We can choose a SQL 
dialect for the generated code. The modus of SQL generation decides how
inheritance relationships are mapped to the SQL schema. In the typed
modus all subclass properties will be written to one table of the super class
(often called generalization table).
Any class has its own table with the own properties in the vertical modus. If
you wish only to generate the code for invariant you must choose \emph{Only
Integrity View}. Otherwise, the table schema will be generated as well. The
other parameter will set the prefix for the different parts of the SQL schema. We can
finish the settings and start the code generation with the \eclipse{Finish} button.
 
\section{The Generated Code}
After finishing the wizard, the code for the selected constraints will be 
generated. To review the results it can be necessary to refresh the
project (F5). Our project contains two new SQL files in the folder \texttt{sql}
(cf. Fig.~\ref{pic:codegen:projectExplorerSQL}). The file 
\texttt{2010-09-29-09-34\_schema.sql} contains the table and view definitions (called object views) of the
model. Every class of the model has its own view. Via this view the
data(objects) of the class can be accesssed. The other file
\texttt{2010-09-29-09-34\_view.sql} contains the SQL code for the invariants (cf.
Fig.~\ref{pic:codegen:generateSQL}), that is one view definition (called integrity view) for each invariant. 
The semantics of such an integrity view is that after evaluation of an invariant the corresponding integrity view contains all objects that
violate the invariant (cf. Listing~\ref{lst:codegen:sqlInvariant}).
 
\begin{figure}[!p]
	\centering
	\includegraphics[width=0.45\linewidth]{figures/codegen/projectExplorerSQL}
	\caption{The Package Explorer containing the new SQL code files.}
	\label{pic:codegen:projectExplorerSQL}

	\vspace{2.0em}
	
	\centering
	\includegraphics[width=0.45\linewidth]{figures/codegen/generateSQL}
	\caption{The Package Explorer containing the new SQL code files.}
	\label{pic:codegen:generateSQL}

	\vspace{2.0em}
	
 \lstset{
    language=SQL
 }
 \begin{lstlisting}[caption={SQL Code for invariant oclInv1.}, captionpos=b,
  label=lst:codegen:sqlInvariant] -- Context: Person
-- Expression: inv tudOclInv1: self.supervisor.grade.value > self.grade.value
CREATE OR REPLACE VIEW tudOclInv1 AS
(SELECT * FROM OV_Person AS SELF
WHERE NOT (((SELECT value FROM OV_Grade
 WHERE PK_Grade IN (SELECT FK_grade FROM OV_Person
 WHERE PK_Person IN (SELECT FK_supervisor FROM OV_Person WHERE PK_Person = SELF.PK_Person))) > (SELECT value FROM OV_Grade
 WHERE PK_Grade IN (SELECT FK_grade FROM OV_Person WHERE PK_Person = SELF.PK_Person)))));
  \end{lstlisting}
\end{figure}

\section{Summary}
 
  
This chapter describes the translation of OCL invariants to SQL integrity views by the 
\keyword{OCL2SQL} code generator of Dresden OCL.
Besides the use of \keyword{OCL2SQL} via Dresden OCL's GUI, you can also invoke
OCL2SQL via Dresden OCL's \acs{API}. The easiest way to connect to Dresden OCL
is via its \emph{Facade} providing interfaces for all services of Dresden OCL. 
How to use Dresden OCL's facade is documented in Chapter~\ref{chapter:integration}.
 