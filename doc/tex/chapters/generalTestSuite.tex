\chapter{The Extensible Test Suite of Dresden OCL}
\label{chapter:generalTestSuite}

\begin{flushright}
\textit{Chapter written by Michael Thiele}
\end{flushright}

Dresden OCL is a collection of Eclipse plug-ins that either are used together or
some of these plug-ins are used together with plug-ins of other parties. This 
modular structure imposes one problem when trying to combine all test cases of 
Dresden OCL into one test suite, since there is no guaranty that all test
plug-ins are available.

Therefore, an extensible test suite has been created. It can be found under the 
name \reference{tudresden.\linebreak[0]ocl20.pivot.testsuite}. Basically it is a
test suite that searches a specific extension point for registered tests or test
suites. It includes these tests in its own test suite and executes the test
suite. Thus, on any change of the source code all tests can be run at once to 
check the integrity of the toolkit.

To extend the extensible test suite with new tests or test suites, a plug-in has
to implement the extension point \reference{org.dresdenocl.testsuite}. 
This extension has to specify the tests it wants to add. \keyword{JUnit3} as
well as \keyword{JUnit4} tests or test suites are allowed. If, for some reason,
the test wants to emit some warnings to the user, it can use the Log4j mechanism
for that purpose. Simply extend the \code{log4j.properties} with the following
code:

\code{\# Extensible Test Suite appender\newline
log4j.appender.stringbuffer=org.dresdenocl.logging.appender.StringBufferAppender\newline
log4j.appender.stringbuffer.layout = org.apache.log4j.PatternLayout\newline
log4j.appender.stringbuffer.layout.ConversionPattern = \%C{1}: \%m\%n\%n}

Then add this appender to the logging of the plug-in. An example usage of this 
mechanism can be found in the plug-in 
\reference{org.dresdenocl.metamodels.test}.

To run the extensible test suite, open the context menu on the class 
\code{OCL2TestSuiteRunner} of the package 
\code{tudresden.\linebreak[0]ocl20.pivot.testsuite.runner} in the 
\eclipse{Package Explorer}. Choose \code{Run As} --\textgreater \code{JUnit 
Plug-in Test} as shown in Figure~\ref{pic:generalTestSuite:RunAs}.

\begin{figure}[!htbp]
	\centering
	\includegraphics[width=1.0\linewidth]{figures/generalTestSuite/RunAs}
	\caption{Run the Extensible Test Suite.}
	\label{pic:generalTestSuite:RunAs}
\end{figure}