\chapter{OCL Interpretation}
\label{chapter:interpretation}

\begin{flushright}
\textit{Chapter written by Claas Wilke}
\end{flushright}

This chapter describes how the \acs{OCL}2 Interpreter provided with DresdenOCL can be used. How to install and run \keyword{Dresden OCL2 for Eclipse} and how to load models and OCL constraints was explained in Chapter~\ref{chapter:introduction}. If you are not familiar with such basic uses of DresdenOCL, read Chapter~\ref{chapter:introduction} first.




\section{The Simple Example}

This Chapter uses the \keyword{Simple Example} which is provided with \acl{DOT4Eclipse} located in the plug-in \reference{tudresden.ocl20.pivot.examples.\linebreak[0]simple}. An overview over all examples provided with \acl{DOT4Eclipse} can be found in Table \ref{tab:examples} in the appendix of this manual. An introduction into the Simple Example can be found in Section \ref{intro:simpleExample}. The model of the example defines three classes: The class \model{Person} has the attributes \model{age} and \model{name}. Two subclasses of \model{Person} are defined, \model{Student} and \model{Professor}.

To import the Simple Example into our Eclipse workspace we create a new Java project called \model{tudresden.ocl20.pivot.examples.\linebreak[0]simple} and use the import wizard \eclipse{General > Archive File} to import the example provided as a jar archive. In the following window we select the directory where the jar file is located (probably the \model{plugins} directory into the Eclipse root folder), select the archive \model{tudresden.ocl20.pivot.examples.simple.\linebreak[0]jar} and push the \eclipse{Finish} button (if you use a source code distribution of \acl{DOT4Eclipse} instead, you can simply import the project \model{tudresden.ocl20.\linebreak[0]pivot.examples.simple} using the import wizard \eclipse{General -> Existing Projects into Workspace}). Figure \ref{pic:example:simple02} shows the \eclipse{Package Explorer} containing the imported project.

\begin{figure}[!p]
	\centering
	\includegraphics[width=0.5\linewidth]{figures/examples/simple02}
	\caption{The Package Explorer containing the Project which is required to run this Tutorial.}
	\label{pic:example:simple02}
	
	\vspace{4.0em}

  \lstset{
    language=OCL
  }
  \begin{lstlisting}[caption={The Constraints contained in the Constraint File.}, captionpos=b, label=lst:interpret:allConstraints]
-- The age of Person can not be negative.
context Person
inv: age >= 0

-- Students should be 16 or older.
context Student
inv: age > 16

-- Proffesors should be at least 30.
context Professor
inv: not (age < 30)

-- Returns the age of a Person.
context Person
def: getAge(): Integer = age

-- Before returning the age, the age must be defined.
context Person::getAge()
pre: not age.oclIsUndefined()

-- The result of getAge must equal to the age of a Person.
context Person::getAge()
post: result = age
  \end{lstlisting}
\end{figure}

The project provides a model file that contains a class diagram (the model file is located at \model{model/simple.uml}) and the constraint file we want to interpret (located at \model{constraints\linebreak[0]/\linebreak[0]all\-Con\-straints.ocl}). Listing \ref{lst:interpret:allConstraints} shows the constraints defined in the constraint file.

First, the constraint file defines three simple invariants that denote, that the \model{age} of every \model{Person} must always zero or greater than zero. Furthermore, the \model{age} of every \model{Student} must be greater than 16 and the  \model{age} of every \model{Professor} does not have to be lesser than 30.

In addition to that the constraint file contains a definition constraint that defines a new operation \model{getAge()} which returns the \model{age} of a \model{Person}. A precondition checks, that the \model{age} must be defined before it can be returned by the operation \model{getAge()}. And finally, a postcondition which checks, whether or not the result of the operation \model{getAge()} is the same as the \model{age} of the \model{Person}.



\section{Preparation of the Interpretation}

To prepare the interpretation we have to import the model \model{model/simple.uml} for which we want to interpret constraints into the \eclipse{Model Browser}. We use the model import wizard of DresdenOCL to import the model. This procedure is explained in Section~\ref{intro:loadModel}. Furthermore, we have to import a model instance for which the constraints shall be interpreted into the \eclipse{Model Instance Browser}. We use another import wizard to import the model instance \model{bin/tudresden/ocl20/pivot/\linebreak[0]exam\-ples/\linebreak[0]simple\linebreak[0]/\linebreak[0]ModelProviderClass.class}. Finally, we have to import the constraint file \model{con\-straints/\linebreak[0]all\-Con\-straints\linebreak[0].ocl} containing the constraints we want to interpret. The import is done by an import wizard again. Afterwards, the \eclipse{Model Browser} should look like illustrated in Figure \ref{pic:interpret:prepare01} and the \eclipse{Model Instance Browser} should look like shown in Figure \ref{pic:interpret:prepare02}.

\begin{figure}[!p]
	\centering
	\includegraphics[width=0.6\linewidth]{figures/interpreter/prepare01}
	\caption{The Model Browser containing the Simple Model and its Constraints.}
	\label{pic:interpret:prepare01}

  \vspace{4.0em}
  
	\centering
	\includegraphics[width=0.6\linewidth]{figures/interpreter/prepare02}
	\caption{The Model Instance Browser containing the Simple Model Instance.}
	\label{pic:interpret:prepare02}
\end{figure}

The opened model instance contains three instances of the classes defined in the Simple Example model. One instance of \model{Person}, one instance of \model{Student} and one instance of \model{Professor}. For these three instances we now want to interpret the imported constraints.



\section{OCL Interpretation}

Now we can start the interpretation. To open the \acs{OCL}2 Interpreter we use the menu option \eclipse{Dresden OCL2 > Open OCL2 Interpreter}. The \eclipse{OCL2 Interpreter View} should now be visible (see Figure \ref{pic:interpret:interpret01}).

By now, the \eclipse{OCL2 Interpreter View} does not contain any result. Besides the results table, the view provides four buttons to control the \acs{OCL}2 Interpreter. The buttons are shown in Figure~\ref{pic:interpret:interpret02}. With the first button (from left to right) constraints can be prepared for interpretation. The second button can be used to add variables to the \keyword{Interpreter's Environment}. The third button provides the core functionality, it can be used to start the interpretation. And finally, the fourth button provides the possibility to delete all results from the \eclipse{OCL2 Interpreter View}. The functionality of the buttons will be explained below.

\begin{figure}[!p]
	\centering
	\includegraphics[width=1.0\linewidth]{figures/interpreter/interpret01}
	\caption{The OCL2 Interpreter View containing no results.}
	\label{pic:interpret:interpret01}

  \vspace{2.0em}
  
	\centering
	\includegraphics[width=0.5\linewidth]{figures/interpreter/interpret02}
	\caption{The Buttons to Control the OCL2 Interpreter.}
	\label{pic:interpret:interpret02}
	
  \vspace{2.0em}

	\centering
	\includegraphics[width=0.6\linewidth]{figures/interpreter/interpret03}
	\caption{The three Invariants selected in the Model Browser.}
	\label{pic:interpret:interpret03}
\end{figure}


\subsection{Interpretation of Constraints}

\begin{figure}[!p]
	\centering
	\includegraphics[width=1.0\linewidth]{figures/interpreter/interpret04}
	\caption{The results of the three Invariants for all Model Instance Elements.}
	\label{pic:interpret:interpret04}

  \vspace{3.0em}

	\centering
	\includegraphics[width=0.4\linewidth]{figures/interpreter/interpret06}
	\caption{The Definition selected in the Model Browser.}
	\label{pic:interpret:interpret06}

  \vspace{3.0em}
  
	\centering
	\includegraphics[width=1.0\linewidth]{figures/interpreter/interpret07}
	\caption{The results of the Definition for all Model Instance Elements.}
	\label{pic:interpret:interpret07}
\end{figure}

To interpret constraints, we simple select them in the \eclipse{Model Browser} and push the button to interpret constraints (the third button from the left). First, we want to interpret the three invariants defining the range of the \model{age} of \model{Persons}, \model{Students} and \model{Professors}. We select them in the \eclipse{Model Browser} (see Figure \ref{pic:interpret:interpret03}) and push the \eclipse{Interpret} button. The result of the interpretation is now shown in the \eclipse{\acs{OCL}2 Interpreter View} (see Figure \ref{pic:interpret:interpret04}).

The invariant \model{age >= 0} has been interpreted for all three model objects. The results for the \model{Person} and the \model{Student} instances are \model{true} because their \model{age} is greater than zero. The result for the \model{Professor} instance is \model{false} because its \model{age} is \model{-42}.

The two other invariants were only interpreted for the \model{Student} or the \model{Pro\-fessor} instance because their context is not the class \model{Person} but the class \model{Student} or the class \model{Professor}, respectively. Again, the \model{Student's} result is \model{true} and the \model{Professor's} result is \model{false}.

Besides invariants, \acs{OCL}2 enables us to use \acs{OCL} expressions to define new attributes and methods or to initialize attributes and methods. Such \model{def}, \model{init} and \model{body} constraints cannot be interpreted to \model{true} or \model{false}, because their result type has not to be \model{Boolean}. Furthermore, they can be used to alter the results of other constraints that shall be interpreted. The \model{allConstraints.ocl} file contains a definition constraint, which defines the method \model{getAge()} for the class \model{Person}.  Now, we want to interpret this definition constraint. We select the constraint in the \eclipse{Model Browser} (see Figure~\ref{pic:interpret:interpret06}) and push the \eclipse{Interpret} button. The result of the interpretation is shown in Figure~\ref{pic:interpret:interpret07}. The interpretation finishes for all three instances successfully because the attribute \model{age} has been set for all three instances.



\subsection{Adding Variables to the Environment}

When interpreting OCL constraints from the GUI, we have to add further context information to interpret some pre- and postconditions. For example, the postcondition contained in the constraint file compares the result of the method \model{getAge()} with the attribute \model{age} of the referenced \model{Person} instance. Therefore, \acs{OCL} provides the special variable \model{result} in postconditions which contains the result of the constrained method's execution. Using the \eclipse{\acs{OCL}2 Interpreter View} we cannot execute the method \model{getAge()} and store the result in the \model{result} variable. We can interpret the postcondition in a specific context which has to be prepared by hand only. We have to set the result variable manually.

If we interpret the postcondition constraint (the sixth and last constraint in the \eclipse{Model Browser}) without setting the \model{result} variable, the constraint results in a \model{undefined} result for all three model instances (see Figure~\ref{pic:interpret:interpret08}).

To prepare the variable, we push the button to add new variables to the Interpreter Environment (the second button from the left) and a new window opens which we can use to specify new variables. We enter the name \model{result}, select the variable type \model{Integer} and enter the value \model{25}. Then we push the \eclipse{OK} button (see Figure \ref{pic:interpret:interpret09}. The result variable has now been added to the Interpreter's Environment.

\begin{figure}[!p]
	\centering
	\includegraphics[width=1.0\linewidth]{figures/interpreter/interpret08}
	\caption{The results of the Postcondition without preparing the Result Variable.}
	\label{pic:interpret:interpret08}

  \vspace{3.0em}
	
	\centering
	\includegraphics[width=0.5\linewidth]{figures/interpreter/interpret09}
	\caption{The Window to add new Variables to the Environment.}
	\label{pic:interpret:interpret09}

  \vspace{3.0em}
	
	\centering
	\includegraphics[width=1.0\linewidth]{figures/interpreter/interpret10}
	\caption{The Results of the Postcondition with Result Variable Preparation.}
	\label{pic:interpret:interpret10}
	
  \vspace{3.0em}

  \lstset{
    language=OCL
  }
  \begin{lstlisting}[caption={An example Precondition defined on an Operation with Argument.}, captionpos=b, label=lst:interpret:precondition]
-- arg01 must be defined.
context Person::setAge(arg01: Integer)
pre: not arg01.oclIsUndefined()
  \end{lstlisting}
\end{figure}

Now, we can interpret the postcondition again. The result is shown in Figure~\ref{pic:interpret:interpret10}. The results for the \model{Student} and \model{Professor} instances are both \model{false} because their \model{age} attribute is not equal to \model{25} and thus the \model{result} value does not match to the \model{age} attribute. But the interpretation for the \model{Person} instance succeeds because its \model{age} is \model{25}.

Other examples requiring manual addition of context information are pre- and postconditions that are defined on operations containing arguments. Listing~\ref{lst:interpret:precondition} shows a precondition that is defined on an operation \code{setAge(arg01)}. If the argument \code{arg01} is referred during interpretation, the interpreter has to know the value of the argument. Thus, we would have to add the value of \code{arg01} before the constraint's interpretation manually as shown for the \code{result} variable.



\subsection{Preparation of Constraints}

The interpretation of some postconditions requires a preparation of the Interpreter's environment before the operation defined in the context of the postcondition is invoked. Listing~\ref{lst:interpret:postcondition} shows such a postcondition. The postcondition is defined on an operation \code{birthdayHappens()} that increments the \code{age} of a \code{Person}. The postcondition checks, whether the \code{age} was incremented or not. Thus, the Interpreter has to store the value of \code{age} before the operation \code{birthdayHappens()} is invoked. Therefore, the Interpreter View provides a button to prepare constraints (the first button from the left). If you want to interpret such postconditions, first select and prepare your constraint. The value of \code{age@pre} is then stored in the Interpreter's environment. Afterwards you can invoke your model instance's operation (which is quite complicate from the GUI). Afterwards you can interpret the postcondition.

\begin{figure}[!t]
  \lstset{
    language=OCL
  }
  \begin{lstlisting}[caption={An example Postcondition that must be prepared.}, captionpos=b, label=lst:interpret:postcondition]
-- age must be incremented by one.
context Person::birthdayHappens()
post: age = age@pre + 1
  \end{lstlisting}
\end{figure}

The preparation of postconditions is not that useful when interpreting constraints from the GUI of DresdenOCL because you cannot invoke your operations here to alter your model instance's state. Nevertheless, like the possibility to add variables to the Interpreter's environment you can prepare postconditions from the GUI. These operations are much more useful when using DresdenOCL via its API and using the \acs{OCL}2 Interpreter to check \acs{OCL} constraints during the runtime of other software. Then you can prepare constraints before methods are invoked and check postconditions afterwards, e.g., by using \emph{\acf{AOP}}.



\section{Summary}
  
This chapter described how \acs{OCL} constraints can be interpreted using the \acs{OCL}2 Interpreter of DresdenOCL. The preparation and interpretation of constraints has been explained, the addition of new variables to the Interpreter Environment has been shown. Besides the use of the Interpreter via DresdenOCL's GUI, you can also invoke the Interpreter via DresdenOCL's \acs{API}. The easiest way to connect to DresdenOCL is via its \emph{Facade} providing interfaces for all services of DresdenOCL. How to use DresdenOCL's facade is documented in Chapter~\ref{chapter:integration}.