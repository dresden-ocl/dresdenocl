Dresden OCL has been developed at the Technische Universit�t Dresden,
Department of Computer Science, Software Technology Group. Dresden OCL and this
manual are available a the Dresden OCL project website
(\url{http://www.dresden-ocl.org/}).



\section*{Contact}

Technische Universit�t Dresden \newline
Fakult�t Informatik\newline
Lehrstuhl Softwaretechnik\newline
Prof. Dr. Uwe A�mann\newline
N�thnitzer Str. 46\newline
D-01187 Dresden



\section*{License}

We are always looking forward to find new projects that use Dresden OCL or at
least parts of it. Thus, please inform us, if you use Dresden OCL wthin your
project, tool or application.


\subsection*{Dresden OCL}

\begin{floatingfigure}[r]{2.0cm}
	\includegraphics[width=2.0cm]{figures/license/gnu.pdf}
\end{floatingfigure}

Dresden OCL is free software: you can redistribute it and/or modify it under the terms of the \keyword{GNU Lesser General Public License} as published by the \keyword{Free Software Foundation}, either version 3 of the License, or (at your option) any later version. Dresden OCL is distributed in the hope that it will be useful, but \textbf{without any warranty}; without even the implied warranty of \textbf{merchantability} or \textbf{fitness for a particular purpose}. See the GNU Lesser General Public License for more details. You should have received a copy of the GNU Lesser General Public License along with Dresden OCL. If not, see \url{http://www.gnu.org/licenses/}.


\subsection*{This Manual}

\begin{floatingfigure}[r]{1.5cm}
	\includegraphics[width=1.5cm]{figures/license/by.pdf}
\end{floatingfigure}

This document is licensed under the \keyword{Creative Commons Attribution 3.0 Unported} license. 
You may share, copy, distribute and transmit this document and you can also adapt this document into your own work. But be aware that you must attribute the work in the manner specified by the author or licensor (but not in any way that suggests that they endorse you or your use of the work). The full license is available under \url{http://creativecommons.org/licenses/by/3.0/}.
