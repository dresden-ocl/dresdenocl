\chapter{The Generic Meta-Model Test Suite}
\label{chapter:metaModelTestSuite}

\begin{flushright}
\textit{Chapter written by Claas Wilke}
\end{flushright}

To test the adaptation of a meta-model to the pivot model of Dresden OCL, the
toolkit provides a generic test suite that can be simply instantiated by each 
adapted meta-model. This chapter shortly presents, how the generic meta-model 
test suite can be instantiated to test an adapted meta-model.



\section{The Test Suite Plug-in}

The generic meta-model test suite is located in the plug-in 
\code{tudresden.ocl20.pivot.meta\-mo\-dels.\linebreak[0]test}. The test suite 
provides a set of JUnit tests, that check the functionality of all operations 
that must be implemented by every meta-model that shall be adapted to the pivot 
model. The adaptation of a meta-model to the pivot model is explained in 
Chapter~\ref{chapter:pivotModelAdaptation}. The test suite contains about 150
Junit tests.

To instantiate the generic test suite for a newly adapted meta-model, only two
resources must be provided: (1) a model modeled in the newly adapted meta-model 
that contains instances of all pivot model types that shall be tested, and (2) a
Java class that instantiates the test suite with the modeled model. During test 
execution, the generic test suite uses the provided model to test the meta-model
(see Figure~\ref{pic:metaModelTestsuite:genericTestSuite}). Both, the model and 
the Java class are shortly presented in the following sections.

\begin{figure}[!t]
	\centering
		\includegraphics[width=0.80\textwidth]{figures/metamodeltestsuite/genericTestSuite.pdf}
	\label{pic:metaModelTestsuite:genericTestSuite}
	\caption{The Generic Meta-Model Test Suite in respect to the Generic Three Layer Ar\-chi\-tec\-ture (as presented in Section \ref{architecture:genericLayers}).}
\end{figure}



\section{The required Model to test a Meta-Model}

Figure~\ref{pic:metaModelTestsuite:testModel} shows the test model that must be
modeled using the meta-model that shall be tested with the generic meta-model 
test suite. At a first sight, the model seems to be very complex. But many of 
the contained features are optional, because some data structures and types of 
the pivot model could be (but do not have to be) implemented by a meta-model. 
E.g., a meta-model can provide an enumeration type but does not have to. If a 
structure is not provided by a test model, the test suite will print a warning 
during test execution that the expected structure has not been found. If the 
structure is not adapted intentionally, the warning can be ignored. In the 
following, all types and relations of the test model are explained shortly.

\begin{sidewaysfigure}[!p]
	\includegraphics[width=0.85\textwidth]{figures/metamodeltestsuite/testModel.pdf}
	\caption{The required Test Model to test a Meta-Model's adaptation. The gray parts are optional.}
	\label{pic:metaModelTestsuite:testModel}
\end{sidewaysfigure}


\subsection{TestTypeClass1 and TestTypeClass2}

As their names already tell us, the classes \code{TestTypeClass1} and 
\code{TestTypeClass2} are used to test the adaptation of \code{Types}. Each 
meta-model has to provide types, thus, these classes are required. Both classes 
provide an operation and a property. The association between the two classes is 
optional because not all meta-models contain associations. But the
generalization between \code{TestTypeClass2} and \code{TestTypeClass1} is 
required.


\subsection{TestTypeInterface1 and TestTypeInterface2}

Besides classes, some meta-models also provide a second type that must be mapped
to the pivot model's interface \code{Type}, which are \code{Interface}s. To test
the adaptation of the \code{Type} element for meta-models that have both,
classes (or types) and interfaces, the test model contains two interfaces 
\code{TestTypeInterface1} and \code{TestTypeInterface2}. They are optional and
can be used to test the adaptation of interfaces. E.g., a meta-model that adapts
both classes and interfaces is the \acs{UML}2 meta-model located in the plug-in 
\code{tudresden.ocl20.pivot.metamodels.uml2}.


\subsection{TestEnumeration}

To test the adaptation of an \code{Enumeration} type, the class
\code{TestEnumeration} can be used. Because enumerations are not part of every 
meta-model, this class of the test model is optional.


\subsection{TestPrimitiveTypeClass}

A special class in the test model is the class \code{TestPrimitiveTypeClass}. 
This class contains a property for each primitive type of the adapted meta-model
that shall be tested. Each property has the type of the \code{PrimitiveType} 
whose adaptation shall be tested. Important is the name of the property. If the 
property's name starts with \code{aBoolean}, the type is tested as adapted to a 
pivot model's \code{PrimitiveType} of the kind \code{Boolean}. If the name 
starts with \code{anInteger} instead, the types is tested as an \code{Integer}. 
E.g., the example property \code{aStringString} shown in 
\code{TestPrimitiveTypeClass} in Figure \ref{pic:metaModelTestsuite:testModel} 
is tested as adapted to a \code{String}. 
Table~\ref{tab:metaModelTestSuite:kindAdaptation} shows the adaptation of the 
different property name prefixes to the different \code{PrimitiveTypeKinds}.

\begin{table}[h]
		\begin{tabular}{|p{7cm}|p{7cm}|}
    \hline
    \textbf{Property Name Prefix} & \textbf{Expected PrimitiveTypeKind} \\
    \hline
    \code{aBoolean...} & \code{Boolean} \\			
    \hline
    \code{anInteger...} & \code{Integer} \\			
    \hline
    \code{aReal...} & \code{Real} \\			
    \hline
    \code{aString...} & \code{String} \\			
    \hline
		\end{tabular}
	\caption{The adaptation of properties' name prefixes to PrimitiveTypeKinds.}
	\label{tab:metaModelTestSuite:kindAdaptation}
\end{table}


\subsection{TestPropertyClass}

The class \code{TestPropertyClass} contains properties to test the right 
adaptation of the pivot model element \code{Property}. Additionally, the class 
has many associations that can be used to test the adaptation of a second
property type for associations (like in the \acs{UML}2 meta-model, plug-in: 
\code{tudresden.ocl20.pivot.metamodels.uml2}). Thus, all associations are 
optional and are not required to test a meta-model's adaptation. The names of 
the contained properties are self-explainable: The property 
\code{nonMultipleProperty} is used to test the adaptation of a \code{Property} 
that cannot contain multiple values. The property \code{staticProperty}
represents a static \code{Property} and is optional because not all meta-models 
contain a \code{static} modifier. The other properties are used to test multiple
properties that are ordered, unordered, unique and non-unique.


\subsection{TestOperationAndParameterClass}

Similar to the class \code{TestPropertyClass} (presented above), the class 
\code{TestOperationAnd\-Pa\-ra\-me\-ter\-Class} contains operations to test the 
adaptation of all different kinds of \code{Operations}. Additionally, the class
is also used to test the adaptation of \code{Parameters} of operations. Some of
the operations are optional (like the static operation and the operation with 
an output value), others are required.


\section{Instantiating the Generic Test Suite}

As mentioned above, to initialize the generic meta-model test suite, only one 
Java class must be implemented that instantiates the test suite with the test 
model modeled useing the adapted meta-model. 
Listing~\ref{list:metaModelTestSuite:constraints01} shows a Java class that 
instantiates the test suite to test the \acs{UML}2 meta-model.

\lstset{
  language=Java
}
\begin{lstlisting}[caption={An instantiation of the generic meta-model test suite.}, captionpos=b, label=list:metaModelTestSuite:constraints01, float]
import tudresden.ocl20.pivot.metamodels.test.MetaModelTestPlugin;
import tudresden.ocl20.pivot.metamodels.test.MetaModelTestSuite;

@Suite.SuiteClasses(value = { MetaModelTestSuite.class })
public class TestUML2MetaModel extends MetaModelTestSuite {

  /** The id of the {@link IMetamodel} which shall be tested. */
  private static final String META_MODEL_ID = UML2MetamodelPlugin.ID;

  /** The path of the model which shall be tested. */
  private static final String TEST_MODEL_PATH = "model/testmodel.uml";

  /**
   * <p>
   * Prepares the {@link MetaModelTestSuite}.
   * </p>
   */
  @BeforeClass
  public static void setUp() {

    MetaModelTestPlugin.prepareTest(UML2MetaModelTestPlugin.PLUGIN_ID, 
      TEST_MODEL_PATH, META_MODEL_ID);
  }
}
\end{lstlisting}

Important is that the class provides a JUnit test suite (according to JUnit 4
conventions), that contains the \code{MetaModelTestSuite} (line 4).
Additionally, the class has to provide a \code{setUp()} method only that can be 
used to setup the test suite before its execution (lines 13 to 23). Inside the 
\code{setUp()} method, the operation
\code{MetaModelTestPlugin.prepareTest(String, String, String)} must be invoked. 
The method initializes the environment of the generic test suite by setting 
three arguments:

\begin{enumerate}
	\item The \code{ID} of the plug-in that contains the test model used for 
	  testing (e.g.,
	  \code{tudresden.ocl20.\linebreak[0]pivot.\linebreak[0]metamodels.uml2.test},
	\item The location of the test model relative to the plug-in's root folder 
	  (e.g., \code{model/testmodel.\linebreak[0]uml}),
	\item And the \code{ID} of the meta-model that shall be tested (e.g. 
	  \code{tudresden.ocl20.pivot.meta\-mo\-dels.\linebreak[0]uml2}.
\end{enumerate}

Afterwards, the implemented Java class can be executed as a \eclipse{JUnit 
Plug-in Test} in Eclipse. The test suite should then inform you (by failed test 
cases) which parts of your meta-model adaptation are wrong implemented or
missing. As mentioned above, warnings caused by missing parts of the test 
model--that were not implemented intentionally--can be ignored.


\section{Summary}

This chapter shortly introduced into the generic meta-model test suite of 
Dresden OCL. For further details of the test suite investigate the test suite
plug-in
\code{tudresden.ocl20.pivot\linebreak[0].me\-ta\-mo\-dels\linebreak[0].test} or
the existing test suite instantiations in the plug-ins \code{tudresden.ocl20.pivot.\linebreak[0]metamodels.uml2.test}, 
\code{tudresden.ocl20.pivot.metamodels.ecore.test}, or 
\code{tudresden.\linebreak[0]ocl\-20.pivot.metamodels.java.test}.
