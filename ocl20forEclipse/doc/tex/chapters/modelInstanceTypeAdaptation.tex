\chapter{Adapting a Model Instance Type to Dresden OCL2 for Eclipse}
\label{chapter:modelInstanceTypeAdaptation}

\begin{flushright}
\textit{Chapter written by Claas Wilke}
\end{flushright}

As mentioned in Chapter \ref{chapter:architecture}, \acl{DOT4Eclipse} is able to interpret \acs{OCL} constraints on different types of model instances. E.g., the same constraints can be interpreted on Java \code{Objects}, \acs{EMF} \code{EObjects} and \acs{XML} files. This is possible because the toolkit abstracts the instance's elements as \code{IModelInstanceElements}. Thus, each type of model instance that shall be connected with the \acs{OCL}2 Interpreter requires its own Model Instance Type Adaptation. How such an adaptation has to be implemented is explained below. First, the different elements that can belong to a model instance type are presented. Afterwards, the \code{IModelInstanceProvider}, \code{IModelInstance} and \code{IModelInstanceFactory} interfaces are explained.


\section{The different types of Model Instance Elements}

Similar to a model, a model instance can have different types of elements. The element types are similar to the different types that can be expressed in models adapted to \acl{DOT4Eclipse}. Figure \ref{pic:modelInstanceTypeAdaptation:typeHierarchy} shows all different types of \code{IModelInstanceElements} that can exist. The different types are explained in the following.

\begin{sidewaysfigure}
	\centering
	\includegraphics[width=1.0\linewidth]{figures/modelInstanceTypeAdaptation/typeHierarchy}
	\caption{The different types of IModelInstanceElements.}
	\label{pic:modelInstanceTypeAdaptation:typeHierarchy}
\end{sidewaysfigure}


\subsection{The IModelInstanceElement Interface}

Each \code{IModelInstanceElement} has to provide a set of methods that is required to handle the adapted objects during interpretation. The methods are shortly explained in the following. Some of these methods are implemented in an abstract \code{IModelInstanceElement} implementation and have not to be implemented. Nevertheless, they are presented for completeness reasons.

\subsubsection{asType(Type)}

The method \code{asType(Type)} is required to cast an element to a given type of its model. E.g., in \acs{OCL} the primitive type \code{Integer} can be casted to \code{Real}. In general, this method should check if the given Type conforms to the adapted element and if so, the result is a new \code{IModelInstanceElement} of the given type. Else an \code{AsTypeCastException} is thrown.

\subsubsection{copyForAtPre()} 

The method \code{copyForAtPre()} is required to create a copy of the element if its value shall be stored during interpretation as an \keyword{@pre-value}. E.g., during the interpretation of the constraint

\code{context Person::birthdayHappens()\\
post: self.age = self.age@pre + 1}

the interpreter has to store the value of the property \code{age}. The value has to be copied, because if \code{age} is incremented during the method's execution, a simple reference would refer to the incremented value\footnote{This is true although age is a primitive type. The integer instance is modeled in Java and thus can be referenced.}. As far as we know, it is rather complicate to copy some objects at runtime - e.g., in Java where the \code{clone()} method is not available for every object. Thus, a \code{CopyForAtPreException} can be thrown, if an element cannot be copied.

\subsubsection{getName()}

The method \code{getName()} returns a string representation of the element. This additional operation exists to provide a different string to display the element in the \acs{GUI} components besides the general \code{toString()} method's result.

\subsubsection{getTypes()}

The method \code{getTypes()} returns a set containing all \code{Types} of the element. In most cases, this set contains exactly one Type. But sometimes, this set can contain multiplye types. This is the case if an \code{IModelInstanceElement} represents an object that inherits from multiple types in the model.

\subsubsection{isKindOf(Type) and isTypeOf(Type)}

The methods \code{isKindof(Type)} and \code{isTypeOf(Type)} are required to check if an \code{IModel\-In\-stance\-Ele\-ment} conforms to a given type or is of a given type, respectively.

\subsubsection{isUndefined()}

The method \code{isUndefined()} checks if an \code{IModelInstanceElement}'s adapted element is \code{null} or not.


\subsection{The Adaptation of Model Instance Objects}

The most important \code{IModelInstanceElement} is the \code{IModelInstanceObject}. It encapsulates the standard objects of an model instance such as a Java \code{Object} or an \acs{EMF} Ecore \code{EObject}. \code{IModelInstanceObject} must be implemented by every model instance type because without this kind of element a model instance type does not do any sense. Besides the inherited methods of \code{IModelInstanceElement} three additional methods have to be implemented. They are explained below.

\subsubsection{getObject()}

The method \code{getObject()} returns the adapted \code{Object} of the \code{IModelInstanceObject}.

\subsubsection{getProperty(Property)}

The method \code{getProperty(Property)} is required to get the property values of an object during the interpretation of \acs{OCL} constraints. The method should return the adapted value of the given property (probably an \code{IModelInstanceCollection} of values if the property is multiple or the instance of \code{IModelInstanceVoid} if the property's value is \code{null}) or throws a \code{Pro\-per\-ty\-Not\-Found\-Ex\-ception} if the given property does not exist. A \code{PropertyAccessException} can be thrown, if an unexpected exception occurs during accessing the object's property.

\subsubsection{invokeOperation(Operation, List<IModelInstanceElement>)}
			
The method \code{invokeOperation(Operation, List<IModelInstanceElement>)} is required to invoke the adapted object's operations during interpretation. The list of arguments (adapted as \code{IModelInstanceElements}) may be empty but not \code{null}. The method should return the adapted value of the operation's invocation (probably an \code{IModelInstanceCollection} of values if the operation is multiple or the instance of \code{IModelInstanceVoid} if the result is \code{null}) or throws an \code{OperationNotFoundException} if the given operation does not exist. A \code{Operation\-Access\-Ex\-ception} can be thrown, if an unexpected exception occurs during invoking the object's operation. This operation is one of the most complicated operations in the complete model instance implementation because it must be able to reconvert adapted model instance elements given as parameters. E.g., a given \code{IModelInstanceObject} has to be unwrapped by calling its \code{getObject()} method. For primitive and collection type implementation instances this is more complicate because they do not contain an adapted object that can be returned. They have to be converted. For details of such a reconvert mechanism investigate the Java implementation that uses \keyword{Java Reflections} to check, whether an operation requires an \code{int}, \code{Integer}, \code{byte}, or \code{Long} (for example) as input.


\subsection{The Adaptation of Primitive Type Instances}

To adapt primitive type instances, the interfaces \code{IModelInstanceBoolean}, \code{IModel\-In\-stance\-In\-te\-ger}, \code{IModelInstanceReal} and \code{IModelInstanceString} exist. Each of them contains an additional method to return the adapted value as a Java Object\footnote{Precisely, a \code{Boolean}, a \code{Long}, a \code{Double} or a \code{String}.}. Because primitive instances do not have a state, they do not have to but can be implemented by a model instance type. Instead of implementing own primitive type instances, the predefined instances \code{JavaModelInstanceBoolean}, \code{JavaModelInstanceInteger}, \code{JavaModelInstanceReal} and \code{JavaModelInstanceString} (located in the plug-in \code{tudresden.ocl20.pivot.modelbus} can be reused.


\subsection{The Adaptation of Collections}

Besides primitive type instances and objects, a collection implementation is required to describe sets of \code{IModelInstanceElements}. The interface \code{IModelInstanceCollection<T extends IModelInstanceElement>} provides three additional methods explained below. The \code{IModel\-In\-stance\-Col\-lection} must not be implemented by every model instance type, the predefined implementation \code{JavaModelInstanceCollection} can be reused instead.

\subsubsection{getCollection()}

The method \code{getCollection()} returns a Java collection containing the \code{IModelInstanceElments} that are contained in the \code{IModelInstanceCollection}.

\subsubsection{isMultiple() and isOrdered()}

The methods \code{isMultiple()} and \code{isOrdered()} identify the different types of \acs{OCL} collections.


\subsection{IModelInstanceEnumerationLiteral}

The interface \code{IModelInstanceEnumerationLiteral} represents instances of \code{Enumerations}. Because enumerations do not have a state, they do not need any adapted \code{Object}. Thus, a standard \code{ModelInstanceEnumerationLiteral} implementation located in the plug-in \code{tudresden.ocl20.\linebreak[0]pivot.modelbus} can be reused. During adaptation, an enumeration literal existing in the model instance just has to be associated to its related \code{EnumerationLiteral} in the instance's model. For details investigate the existing \code{IModelInstance} implementations for Java and \acs{EMF} Ecore.


\subsection{IModelInstanceTuple}

The interface \code{IModelInstanceTuple} represents key (\code{IModelInstanceString}) value \code{IModel\-In\-stance\-Ele\-ment} data structure called \keyword{Tuple}. Tuples are required during \acs{OCL} interpretation only. Thus, a standard \code{ModelInstanceTuple} implementation located in the plug-in \code{tudresden.ocl20.\linebreak[0]pivot.modelbus} exists.


\subsection{IModelInstanceVoid and IModelInstanceInvalid}

The interfaces \code{IModelInstanceVoid} and \code{IModelInstanceInvalid} exist to define the singleton instances of the types \code{OclVoid} and \code{OclInvalid}. Their instances can be accessed via the static property \code{IModelInstanceVoid.INSTANCE} or \code{IModelInstanceInvalid.INSTANCE} respectively. For example, the \code{IModelInstanceVoid} instance is required when a method's invocation shall return a \code{null} value.



\section{The IModelInstanceProvider Interface}

Besides the \code{IModelInstaceElements}, a model instance type has to implement an \code{IModel\-In\-stance\-Pro\-vider} that has to be registered at the model-bus plug-in via the extension point \code{tu\-dres\-den\linebreak[0].ocl20.pivot.modelbus.modelinstancetypes}. The model instance provider provides the methods to load a resource (given as a \code{URL} or \code{File}) into an \code{IModelInstance} object. You can use the abstract implementation \code{AbstractModelInstanceProvider} to implement your model instance provider. The two remaining methods to be implemented are explained below.


\subsection{getModelInstance(URL, IModel)}

The method \code{getModelInstance(URL, IModel)} is responsible to load a given model instance (as a \code{URL}) as an instance of a given model. For implementation details investigate the existing implementations for Java and \acs{EMF} Ecore.


\subsection{createEmptyModelInstance(IModel)}

The method \code{createEmptyModelInstance(IModel)} can be used to create an empty model instance for a given model. The model instance can be enriched with objects during runtime via the method \code{IModelInstance.addModelInstanceElement(Object)}.



\section{The IModelInstance Interface}

\begin{figure}
	\centering
	\includegraphics[width=0.8\linewidth]{figures/modelInstanceTypeAdaptation/modelInstanceInterface}
	\caption{The IModelInstance Interface.}
	\label{pic:modelInstanceTypeAdaptation:modelInstanceInterface}
\end{figure}

Figure \ref{pic:modelInstanceTypeAdaptation:modelInstanceInterface} shows the interface \code{IModelInstance}. Many of its operations are implemented in the abstract basis implementation \code{AbstractModelInstance}. The remaining operations that must be implemented are explained below.


\subsection{The Constructor}

The most important operation of a model instance is the constructor. Inside the constructor the resource given to the \code{IModelInstanceProvider} is opened and adapted to \code{IModel\-In\-stance\-Ele\-ments}. To adapt the elements, an \code{IModelInstanceFactory} (explained below) is used. For details investigate the existing \code{IModelInstance} implementations for Java and \acs{EMF} Ecore.


\subsection{addModelInstanceElement(IModelInstanceElement)}

The method \code{addModelInstanceElement(IModelInstanceElement)} can be used to add another \code{Object} to the model instance during runtime. The implementation should use its \code{IModel\-In\-stance\-Fac\-tory} to adapt the \code{Object} and throw a \code{TypeNotFoundInModelException} if the given \code{Object} can not be adapted to the model instance.


\subsection{getStaticProperty(Property)}

The method \code{getStaticProperty(Property)} is required to get the property values of static properties during the interpretation of \acs{OCL} constraints. The method should return the adapted value of the static property (probably an \code{IModelInstanceCollection} of values if the property is multiple or the instance of \code{IModelInstanceVoid} if the property's value is \code{null}) or throws a \code{PropertyNotFoundException} if the given property does not exist. A \code{PropertyAccessException} can be thrown, if an unexpected exception occurs during accessing the object's property.


\subsection{invokeStaticOperation(Operation, List<IModelInstanceElement>)}
			
The method \code{invokeStaticOperation(Operation, List<IModelInstanceElement>)} is required to invoke static operations during interpretation. The list of arguments (adapted as \code{IModel\-In\-stance\-Ele\-ments}) may be empty but not \code{null}. The method should return the adapted value of the operation's invocation (probably an \code{IModelInstanceCollection} of values if the operation is multiple or the instance of \code{IModelInstanceVoid} if the result is \code{null}) or throws an \code{Operation\-Not\-Found\-Ex\-ception} if the given operation does not exist. An \code{OperationAccessException} can be thrown, if an unexpected exception occurs during invoking the object's operation. This operation is one of the most complicated operations in the complete model instance implementation because it must be able to reconvert adapted model instance elements given as parameters. E.g., a given \code{IModelInstanceObject} can be unwrapped by invoking its \code{getObject()} method. For primitive and collection type implementations this is more complicate because they do not contain an adapted object that can be simply returned. They have to be converted. For details of such a reconvert mechanism investigate the Java implementation that uses \keyword{Java Reflections} to check, whether an operation requires an \code{int}, \code{Integer}, \code{byte}, or \code{Long} (for example) as input.



\section{The IModelInstanceFactory Interface}

\begin{figure}
	\centering
	\includegraphics[width=1.0\linewidth]{figures/modelInstanceTypeAdaptation/modelInstanceFactoryInterface}
	\caption{The IModelInstance Interface.}
	\label{pic:modelInstanceTypeAdaptation:modelInstanceFactoryInterface}
\end{figure}

The \code{IModelInstanceFactory} implementation of a model instance is responsible to adapted the instance's objects to the \code{IModelInstanceElement} implementations. It investigates the types of the objects to decide if they shall be adapted as \code{IModelInstanceObjects}, \code{IModelInstance\-Pri\-mi\-tive\-Types} or \code{IModelInstanceEnumerationLiterals}. An \code{IModelInstanceFactory} has to implement four methods as shown in Figure \ref{pic:modelInstanceTypeAdaptation:modelInstanceFactoryInterface}. A default implementation for the basis elements such as \code{IModelInstanceTuples} and \code{IModelInstanceCollections} called \code{BasisJavaModel\-In\-stance\-Fac\-tory} exists. Normally, an \code{IModelInstanceFactory} should extend the \code{BasisJava\-Mo\-del\-In\-stance\-Factory} and should call the methods of the basis implementation as often as possible (e.g., to adapt the \code{IModelInstanceTuples}). For details investigate the existing implementations for \acs{EMF} Ecore and Java.



\section{Adapting an own Model Instance Type}

We know that adapting a model instance type sounds easy but can be a lot of pain. Every kind of model instance comes with its own problems and own solutions. Some may be simple, others may be complicate or impossible. But never forget, if you adapted your own type of model instance, you can connect your instances with \acl{DOT4Eclipse} and you can reuse the \acs{OCL}2 Parser and \acs{OCL}2 Interpreter! 

If you are confused and still do not know how to adapt your model instance type, investigate the existing adaptations for Java (\code{tudresden.ocl20.pivot.modelinstancetype.java}) and \acs{EMF} Ecore (\code{tudresden.ocl20.pivot.modelinstancetype.ecore}). To check your adaptation, have a look at the \keyword{Generic Model Instance Type Test Suite} (presented in Chapter \ref{chapter:modelInstanceTestSuite}) as well.