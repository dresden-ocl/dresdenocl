\section{Conclusion}
\label{sec:conclusion}

In this paper we presented a generic approach for OCL
interpretation that addresses both model and
model instance variability. Various OCL infrastructures support 
model variability, whereas -- to the best of our knowledge -- none of 
the existing OCL infrastructure supports complete model instance
variability. Our approach addresses this problem by
abstracting from domain-specific concepts and by introducing well-defined interfaces for models
and their instances. With our implementation of such a
generic adaptation architecture, the same OCL interpreter was applied to
three case studies that are located at different mo\-del\-ling layers and
use different combinations of models and model instances. We
avoided new implementations of the OCL standard library for various different
technical spaces and hence contribute a reusable OCL interpreter.

For future work, we plan to improve our approach by addressing the issues mentioned 
in Sect.~\ref{sec:lessons}. We are interested in evaluating
the performance impact of our adapter-based approach for OCL interpretation.
Therefore, we plan a benchmark comparing our interpreter with other
interpreters and compilers, and a continuation of our previous
work~\cite{OCLRelDB} on extensible OCL compilation.


\section{Acknowledgements}

We want to thank Tricia Balfe of \textsc{Nomos Software} for providing data for the XML case study and for continuous feedback during adaptation of the case study.
Furthermore, we would like to thank all people that are or were involved in the DresdenOCL project.