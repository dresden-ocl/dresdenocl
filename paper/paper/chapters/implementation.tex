\section{Implementation}
\label{sec:implementation}
	In this section we discuss the implementation of a \emph{Generic Adaptation
	Architecture} to realize the
	variation points identified in the previous section. First, we present
	\emph{Model Adaptation} to address VP1. Afterwards, we dicsuss how \emph{Model
	Instance Adaptation} realises the second variation point (VP2).
	
	\begin{figure}[!t]
			\centering
				\includegraphics[width=0.80\textwidth]{figures/coreconcepts.pdf}
			\caption{
			Interfaces for model and model instance adaptation
% 			The core concepts of the \emph{Model Types} and \emph{Model Instance Types}:
% 			Each \emph{Model} has a root \emph{Namespace} that contains a set of nested Namespaces and 
% 			a set of \emph{Types}. Each Type has a set of \emph{Operations} 
% 			and \emph{Properties}. 
% 			Each \emph{Model Instance} has a set of \emph{Model Instance Elements}. Each Model Instance
% 			Element has exactly one Type and provides 
% 			operations to reflect on this Type. \emph{Model Instance Objects} provide further reflective 
% 			operations to get properties and to invoke operations.
			}
			\label{fig:coreconcepts}
		\end{figure}

\subsection{Model Adaptation}

	\begin{figure}[!t]
			\centering
				\includegraphics[width=1.00\textwidth]{figures/modeladaptation.pdf}
			\caption{The \emph{Generic Adaptation Architecture} of DresdenOCL
% 			: At the Mn+1 layer, meta-models are adapted
% 			to the model types (VP1). The model adapter component contains these adapters and is responsible 
% 			to instantiate them to adapt models of the meta-model. 
% 			At the Mn layer, model implementation types are adapted (VP2). The model instance adapter component
% 			contains these adapters and is responsible to instantiate them to adapt model instance objects. 
% 			The OCL standard library implements the logic to evaluate operations defined on OCL types. Other 
% 			requests such as operation invocations or property requests on adapted objects are delegated 
% 			via the interfaces of the model implementation type model.
			}
			\label{fig:modeladaptation}
		\end{figure}

	To enable the definition of OCL constraints for various modelling languages,
	DresdenOCL provides a set of common interfaces abstracting structures
	that are required to navigate and query object-oriented models.
	These interfaces -- called \emph{Model Types} (or \emph{Pivot Model}) \cite{braeuerOCL07} -- standardize
	the basic	concepts such as types, properties, operations and parameters
	that bind OCL constraints to a concrete modelling language (cf. Fig. \ref{fig:coreconcepts}).
	DresdenOCL only	uses these concepts to parse and statically analyse OCL constraints, 
	e.g., the OCL2 parser	invokes the operation \texttt{getType()} to access the \texttt{Type} of 
	an \texttt{Operation} or \texttt{Property}.
	
	For every \remove{meta-}model that shall be connected with DresdenOCL, 
	a \emph{Model Adapter} component has to be implemented (cf. Fig. \ref{fig:modeladaptation}, Mn+1 layer). 
	It contains individual adapters that map concepts of the modelling
	language to corresponding artefacts of the model types. E.g., the UML
	meta-model concept \texttt{UMLClass} is adapted to the model type concept
	\texttt{Type}. 
	Furthermore, the model adapter component has to provide a factory to create 
	adapters on demand resulting in an \textit{Adapted Model} (cf. Fig.
	\ref{fig:modeladaptation}, Mn+1 and Mn layer).
	The adapters are only created for model elements that are required and
	existing adapters are cached. Thus, unnecessary and expensive adaptation is avoided, 
	especially when working on large models of which only parts are constrained using OCL.
% 	The presented architecture provides a generic and easy variation of multiple meta-models
% 	and corresponding models in an OCL infrastructure (VP1).


\subsection{Model Instance Adaptation}
	
% 	The presented abstraction of model types to support various models
% 	and meta-models can now be shifted and extended to
% 	solve similar problems when working with multiple model instantiations. 
	To realise variation point (VP2), our Generic Adaptation Architecture
	re-applies the adaptation pattern for model instance adaptation.
	Similar to the adaptation of different models, we hide model
	instances behind
	interfaces. This enables the reuse of the same OCL interpreter for the dynamic
	evaluation of OCL constraints on model instance in various implementation
	infrastructures. 
	
	\note{What is the difference between a model instance type and a
	modelInstanceElement? Confusing. Revise and check for overlaps with previous
	sect.} To provide means for model instance adaptation, we introduced the
	\emph{Model Instance Types}. \note{Michael: I don't get the next sentence} The model instance types are different
	interfaces for instances of standard types in OCL such as primitive types, 
	collections and objects (cf. Fig. \ref{fig:coreconcepts}). 
	All these	interfaces inherit a common interface \texttt{ModelInstanceElement}. The most 
	important difference between the model types and the model instance types
	is that \texttt{ModelInstanceElements} provide a \emph{Reflection} \cite{maesOOPSLA87} mechanism whereas 
	model type elements only allow to reason on relationship between concepts of
	the same meta-level. During interpretation, the OCL2 interpreter uses these reflection mechanisms to
	retrieve the \texttt{Type} of a \texttt{ModelInstanceElement}, access \texttt{Property}
	values, or to invoke \texttt{Operations}.
	
	Each kind of model instance that shall be connected with DresdenOCL is
	adapted via a \emph{Model Instance Adapter} component (cf. Fig.
	\ref{fig:modeladaptation}, Mn layer). The model instance adapter component contains adapters that map elements of a
	concrete model instance to model instance types and has to
	provide a factory that creates \emph{Model Object Adapters} for the runtime 
	objects of the adapted model instance (cf. Fig. \ref{fig:modeladaptation}, Mn-1 layer). 
	Like the model adapter, the model instance adapter also 
	creates adapters for objects on demand. Adapted objects are cached to
	improve the performance and to avoid phenomena like \emph{Object
	Schizophrenia} \cite{assmann:isc}.\note{Claas: Reference Okay? Christian:
	Didn't florian send the original source? Always use the most concrete you know}
	
	% said a few times before 
	%
	%Due to the introduction of a common set of model instance types it is possible
	%to easily reuse the same OCL interpreter for various kinds of model
	%instances. Thus, VP2 can be completely addressed by the presented generic
	%adapter architecture.


