\section{Case Studies}

In this section we \remove{shortly} present three case studies \remove{we 
used to investigate}\add{demonstrate} the power of our \note{do we have a
generic interpretation approach or a adaptation approach? What is the
name of the game? Only on approach per paper :) }{generic interpretation
approach}. \remove{Please be aware of the fact that the case studies are
examples for adaptation only. Multiple other adaptations are possible and tasks
for future works.}


\subsection{The Royal and Loyal System Example}

As a first case study, we modeled and implemented the \textit{Royal and Loyal System Example} 
as defined in \cite{warmer:ocl}. The case study was desigend by Warmer and Kleppe to explain 
the Object Constraint Language and can be used to test the interpretation of
\change{almost all possible}{various} kinds of OCL expressions. The case study
consists of 13 UML classes (including inheritance and enumeration types)
\add{and 130 constraints}. We modeled the Royal and Loyal System \remove{by}
using the UML2 meta-model of the Eclipse Model Development Tools (MDT) project
\cite{WWW:MDT}. \remove{We took over 130 constraints from the book}
\add{The model was implemented and instantiated in Java. Consequently
constraints are evaluated on Java objects.}

\begin{figure}[tb]
	\centering
		\includegraphics[width=1.00\textwidth]{figures/casestudy01.pdf}
	\caption{In the Royal and Loyal case study, the MTD UML2 meta-model is adapted at M2. At M1, the Royal and Loyal UML class diagram is adapted as a model and its Java classes are adapted as a model implementation. At M0, each object of the Java classes is adapted as a model instance element.}
	\label{fig:casestudy01}
\end{figure}

\note{revise next paragraph. Differentiate what belongs to model and model
instance adaptation} To \change{load}{pares} the Royal and Loyal \change{System Example}{constraints}
in DresdenOCL, the MDT UML2 meta-model was adapted by the
\note{concreteNameFromFigure} meta-model adapter \add{at the layer M2}
(cf.~Fig. \ref{fig:casestudy01}). \remove{Here, the meta-model adaptation was
realized at the layer M2}. At M1, the class diagram describing the Royal and
Loyal System Example was adapted as an \texttt{IModel}. Furthermore, the 
Java implementation was adapted to an implementation of this model. E.g., the
Java class \texttt{LoyaltyAccount} was mapped to an implementation of the UML
class \texttt{LoyaltyAccount}. \remove{Please be aware of the fact, that} Both
classes are located at the M1 layer because the Java class is only another representation 
of the same class as described by the UML class. 
To load the Java classes as a
model implementation, we had to implement a model implementation adapter for 
Java. This adapter was also responsible to adapt the Java objects to 
\texttt{IModelInstanceElements} at the M0 layer. The Royal and Loyal case 
study proved a sound implementation of the OCL Standard Library specification 
and also a correct adaptation of the MDT UML meta-model and Java model instance objects.



\subsection{SEPA Business Rules}
\note{short intro: similar to prev. sect.}

The company Nomos Software provides a service to check business rules on
financial SEPA \note{introduce abbrev.} messages that are \change{required
for}{used in} financial transactions of bank offices as defined by the European
Payment Council (EPC), ISO20022, and the Euro Banking Association (EBA) \cite{spec:UNIFI,spec:EPC}. 
SEPA messages are described in XML documents that are instances of \add{an} XML
Schema Definitions (XSD). Nomos Software uses OCL constraints defined on such
schema definitions to validate XML documents against a set of business rules
\note{what do these roles constraint?}

\begin{figure}[tb]
	\centering
		\includegraphics[width=1.00\textwidth]{figures/casestudy02.pdf}
	\caption{In the SEPA case study, the XSD meta-model is adapted at M2. At M1, the Pain XML Schema is adapted as a model. The types of XML nodes are mapped to the types in the XSD file. At M0, each node of the XML instance is adapted as a model instance element.}
	\label{fig:casestudy02}
\end{figure}

We implemented an XSD meta-model and an XML model implementation adaptation 
for DresdenOCL and used the \remove{same} XSD and XML files that are 
provided with an online demo of the Nomos service\footnote{Available at http://www.nomos-software.com/demo.html} 
to test our adaptation (see Figure \ref{fig:casestudy02}). At the M2 layer, 
the XML Schema meta-model was adapted by a meta-model adapter. At the M1 layer, 
the \texttt{pain.008.001.01.xml} was adapted as an \texttt{IModel}. An 
adaptation for XML documents as model implementations was realized as well. 
Each type of a node in the XML document was mapped to a Type defined in the 
XML Schema at layer M1. At layer M0, each node was adapted as an 
\texttt{IModelInstanceElement}. We took the \remove{same} about 120 constrains
provided by Nomos Software to test our adaptation. These constraints were
evaluated for three different XML files and the results have been
\add{successfully} compared with the results of the Nomos service demo\remove{to
check the right interpretation of the constraints}. This shows that our model
implementation adaptation allows DresdenOCL to transparantly interpret
constraints on XML as well. The OCL2 Interpreter had not to be
modified\remove{to interpret OCL constraints on XML nodes}.


\subsection{The OCL2.2 Standard Library}
The last case study depicts the ability to load different model instances of one model 
in order to check for inconsistencies between both instances. In this example we 
check WFRs for the OCL standard library of Dresden OCL. 
DresdenOCL's standard library is explicitly modelled as 
an instance of the PivotModel, describing predefined OCL types like \texttt{Integer}, 
\texttt{OclAny} or \texttt{Sequence} and their associated operations. 
Hence, accessing predefined OCL types is reduced to a simple model 
import while the model can conveniently be queried, validated or altered 
\cite{braeuerOCL07}. The WFRs can be used to check whether all OCL types are 
declared and whether they support all operations that are defined by the 
current OCL standard.

Although modelling the standard library leads to great flexibility, the standard library 
still needs an implementation that provides its dynamic semantics. 
This implementation is provided through Java code. As the PivotModel 
does not support code generation, this can lead to inconsistencies between 
the modelled standard library and the according Java interfaces.
We propose to use OCL to check that all modelled types have an equivalent Java 
interface and all modelled operations are also present in the Java interface. Furthermore, 
this approach allows us to check whether the standard library supports all operations that 
are defined in the current OCL standard.

In order to evaluate the WFRs, the PivotModel needs to be loaded as a model (see Figure 
\ref{fig:casestudy03}) via the Ecore Model Adapter. Now, the modelled standard library 
can be loaded as an instance of the PivotModel via the XMI Model Instance Adapter. Thus, 
we are able to evaluate the WFRs for the standard library model and can prove the correct 
structure of it according to the current OCL standard. 

In order to check for inconsistencies with the Java implementation, the Java interfaces 
for predefined OCL types have be loaded with the Java Class Model Instance Adapter. Then, the same WFRs 
used for the standard library model before can be checked for this instance. 
Unfortunately, the Java Class Model Instance Adapter does not exist yet, but will be 
implemented in the near future.


\subsection{Future Work}

For future case studies we plan to adapt meta-model and model implementations for web services (Meta-Model: WSDL, Model Implementation: TODO), static programming languages such as C\# (Meta-Model: UML2, Model Implementation: C\#), data bases (Meta-Model: SQL-DDL, Model Implementation: SQL). 

\textbf{TODO} WFRs of OCL SL ...

